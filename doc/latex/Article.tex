\documentclass{disser}

\usepackage[english,russian]{babel}
\usepackage[utf8x]{inputenc}
\usepackage{amsmath}

\begin{document}

\begin{abstract}
Статья рассматривает статистический метод моделирования быстродействия программ на компьютерах общего назначения \textit{Velocitas}. Дано описание метода. Произведена программная реализация инструментария моделирования быстродействия программ \textit{Adaptor} включающего в себя данный метод. Произведена оценка эффективности метода \textit{Velocitas}. Достигнута более высокая по сравнению с аналогами точность предсказания быстродействия.
\end{abstract}

\section*{Введение}
\subsection*{Обзор области}
В ходе проектирования компьютерных систем часто возникает задача оценки быстродействия определённых программ на данной системе. Традиционным методом решения данной задачи является эмуляция исполнения программы \cite{emulation}. У этого метода есть несколько недостатков. Во-первых, он требует полной реализации поведения эмулируемого компьютера, что требует больших вложений ресурсов \cite{emulation-complexity}. Во-вторых, скорость выполнения программы на эмулируемом компьютере меньше скорости реального выполнения в сотни и тысячи раз \cite{emulation-speed}.
\subsection*{Актуальность}
Как уже было отмечено выше, одной из областей применения моделирования компьютерных систем является оценка быстродействия ещё не произведённого компьютера на этапе его разработки.

Другой пример "--- оценка быстродействия программы при итеративной компиляции.
\section{Методология моделирования, применяемая в системе \textit{Adaptor}}
\subsection{Метод моделирования быстродействия программ \textit{Velocitas}}

Метод \textit{Velocitas} удовлетворяет требованиям к практически применимому решению задачи моделирования быстродействия программ. Основные положения метода описаны ниже.

В первую очередь устанавливается, какие атрибуты объектов в наборе данных являются признаками (входными параметрами), а какой "--- откликом (выходным параметром). Примерами признаков являются характеристики аппаратного обеспечения, на котором запускалась программа, а отклика "--- время исполнения программы. Выбор выходного параметра, то есть критерия эффективности исполнения программы, из присутствующих в наборе данных, осуществляется исследователем.

После этого создаются дополнительные признаки из уже присутствующих в наборе данных. Например, может быть создан признак <<размер входных данных программы>>, определяемый как произведение числа строк и столбцов в матрице, обрабатываемой программой. Число и содержание дополнительных признаков определяется исследователем на основании личного опыта.

Затем производится фильтрация данных с целью удаления заведомо некорректных измерений. Так, шумом являются все опыты, в результате которых для времени исполнения программы получено неадекватно маленькое значение "--- менее $0,001 \textup{с}$ (это минимально измеримое системой время исполнения). Критерии фильтрации также определяются исследователем. Обычно это является элементарной задачей.

Следующим этапом является выполнение ранжирования признаков, присутствующих в наборе данных. Ранжирование выполняется с помощью метода \textit{Earth Importance} \cite{earth-importance}, который является свободной реализацией метода \textit{MARS} "--- \textit{Multivariate Adaptive Regression Splines}. С помощью метода производится оценка важности признаков, после чего для использования при обучении выбираются только признаки с ненулевой важностью.

Построение модели заканчивается выполнением обучения регресионного предсказателя на базе \textit{k Nearest Neighbours}. Квази-оптимальным значением параметра является $k = 30$, однако оно может быть и во многих случаях должно быть настроено индивидуально с учётом имеющегося набора данных. Это можно выполнить с помощью оптимизации гиперпараметров путём сеточного поиска (англ. \textit{grid search}, \cite{grid-search}).

\subsection{Выводы}
Предложен метод моделирования быстродействия программ на различных программно-аппаратных платформах при различных входных данных. Он является удобным для практического применения в инструментарии моделирования и отличается высокими показателями эффективности (см. раздел \ref{sec:evaluation}).

\subsection{Метод измерения времени исполнения исследуемой программы}

\section{Программная реализация}
\subsection{Архитектура. Общая информация}

\subsection{Подсистемы сервера}

\subsection{Подсистемы клиента}

\section{Исследование эффективности}
\subsection{Методология и экспериментальное окружение}

\subsection{Анализ результатов}

\section*{Заключение}

\end{document}