\chapter{Проверка оптимизаций, сохраняющих структуру программы}

Опишем стратегию \emph{voc} и теорию валидации оптимизаций.

Компилятор получает на вход \emph{исходную программу}, написанную на высокоуровневом языке, преобразует её в \emph{промежуточное представление} и затем применяет набор оптимизаций к программе --- начиная с классических глобальных оптимизаций, независящих от архитектуры, и заканчивая архитектурно-зависимыми, такими, как распределение регистров и планирование инструкций. Обычно эти оптимизации производятся в несколько проходов (до 15 в некоторых компиляторах), и каждый проход применяет оптимизацию определённого типа.

Для того, чтобы доказать, что целевой код является корректной трансляцией исходного кода, разберём сначала некоторые необходимые термины. Целевой код \emph{T} является корректной трансляцией исходного кода \emph{S}, если вычисление кода \emph{T} с некоторыми значениями переменных эквивалентно вычислению кода \emph{S} с некоторыми другими значениями переменных.

Промежуточное представление --- это трёхадресный код. Он описан \emph{графом потока}, что является графовым представлением трёхадресного кода. Каждый узел в графе потока представляет собой \emph{линейный участок}, то есть последовательность операторов, не содержащую ветвлений. Границы линейного участка определяются потоком управления.

\section{Системы преобразований}

Для представления формальных понятий исходного и промежуточного кода вводится понятие \emph{системы преобразований} (англ. transition systems, $TS$), вариант аналогичного понятия, упомянутого в \cite{PSS98b}. Система преобразований $S = \left\langle  V, O, \Theta, \rho \right\rangle $ --- конечный автомат, состоящий из:
\begin{itemize}
\item $V$ --- множество переменных состояния,
\item $O \subseteq V$ --- множество наблюдаемых переменных состояния,
\item $\Theta$ --- начальное состояние системы,
\item $\rho$ --- правила перехода, сопостовляющие исходное и результирующее состояние.
\end{itemize}

Значение, получаемое переменной $x$ в состоянии $s$, обозначается $s[x]$. Правило перехода $y' = y + 1$ означает, что в новом состоянии значение переменной $y$ на 1 больше, чем в старом.

Наблюдаемыми считаются все переменные, значение которых выводится на внешние устройства в результате работы программы.

Система переходов называется детерминированной, если начальное состояние однозначно определяет дальнейший ход вычислений. Мы рассматриваем валидацию только таких программ.

Определим $P_{s} = \left\langle  V_{s}, O_{s}, \Theta_{s}, \rho_{s} \right\rangle $ и $P_{t} = \left\langle  V_{t}, O_{t}, \Theta_{t}, \rho_{t} \right\rangle $ как исходную и целевую систему переходов, соответственно. Они называются сравнимыми, если существует соотношение один-к-одному между наблюдаемыми переменными $P_{s}$ и $P_{t}$. Обозначим $ X \subset O_{s} $ и $ x \subset O_{t} $ соответствующие наблюдаемые переменные. Исходное состояние \emph{совместимо} с целевым, если они оба включают одинаковые множества наблюдаемых переменных. Назовём $P_{t}$ корректной трансляцией $P_{s}$ если они сравнимы и, для каждого вычисления $P_{T}$ $\sigma_{T}: t_{0}, t_{1}, \ldots$ и $P_{S}$ $\sigma_{S}: s_{0}, s_{1}, \ldots $ $s_{0}$ совместимо с $t_{0}$, $\sigma_{T}$ завершается тогда и только тогда, когда $\sigma_{S}$ завершается, и их конечные состояния совместимы.
