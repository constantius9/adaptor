\section*{Введение}
\addcontentsline{toc}{section}{Введение}%

В последнее время понимание необходимости формального доказательства правильности работы критических частей систем широко распространено как в научной среде, так и в промышленности. Большинство методов проверки нацелено на проверку соответствия спецификации требованиям, и проверке соответствия высокоуровнего кода спецификации. Однако, если мы хотим доказать, что высокоуровневая спецификация правильно реализована в низкоуровневом коде, нам надо удостовериться в правильности работы компилятора, производящего оптимизации. Проверка правильности современного оптимизирующего компилятора --- сложная задача, потому что целевая архитектура сложна и имеет гибкую конфигурацию, а также из-за сложных алгоритмов анализа и оптимизации, используемых в компиляторах.

Формальная проверка полномасштабного оптимизирующего компилятора таким же образом, каким мы бы проверяли любую другую большую программу, не осуществима из-за его размера и, в ряде случаев, закрытого исходного кода компилятора. \emph{Проверка трансляции} --- новый подход, который предлагает альтернативу проверке трансляторов в общем и компиляторов в частности. Этот метод предполагает предварительное создание \emph{специального инструмента}, который формально проверяет правильность преобразования программы после каждого прохода компилятора.

Основная цель современной формальной проверки компиляторов --- обеспечить поддержку широкого круга оптимизаций, в том числе и агрессивных, меняющих структуру программы, таких, как распределение циклов, слияние циклов и обмен циклов. Текущие работы \cite{PSS98a,Nec00,RM00,ZPFG02} до сих пор не рассматривали настолько полный анализ правильности работы компилятора.

В реферате по очереди рассмотрено несколько тем. Сначала мы поговорим о проверке классических оптимизаций, не меняющих структуру программы. Затем мы рассмотрим проверку более агрессивных оптимизаций, меняющих структуру, после чего обсудим ограничения статической проверки и способы преодоления этих ограничений с помощью динамической проверки правильности.