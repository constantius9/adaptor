\section{Введение}
В последнее время понимание необходимости формального доказательства правильности работы критических частей систем широко распространено как в научной среде, так и в промышленности. Большинство методов проверки фокусируются на проверке соответствия спецификации требованиям, и проверке соответствия высокоуровнего кода спецификации. Однако, если мы хотим доказать, что высокоуровневая спецификация правильно реализована в низкоуровневом коде, нам надо удостовериться в правильности работы компилятора, производящего оптимизации. Проверка правильности современного оптимизирующего компилятора --- сложная задача, потому что целевая архитектура сложна и имеет гибкую конфигурацию, а также из-за сложных алгоритмов анализа и оптимизации, используемых в компиляторах.

Формальная проверка полномасштабного оптимизирующего компилятора таким же образом, как мы бы проверяли любую другую большую программу, не осуществима из-за его размера и, возможно, проприетарных соображений. \emph{Валидация трансляции} --- новый подход, который предлагает альтернативу проверке трансляторов в общем и компиляторов в частности. В этом подходе сначала создаётся \emph{инструмент валидации}, который формально проверяет правильность преобразования программы после каждого запуска компилятора.

Основная цель современной формальной проверки компиляторов --- обеспечить поддержку широкого круга оптимизаций, в том числе и агрессивных, меняющих структуру программы, таких, как распределение циклов, слияние циклов и обмен циклов. Текущие работы \cite{PSS98a,Nec00,RM00,ZPFG02} до сих пор не рассматривали настолько полный анализ правильности.

В реферате рассматривается проверка оптимизаций, не меняющих структуру программы, затем проверка оптимизаций, меняющих структуру, после чего рассматриваются ограничения статической проверки и их обход с помощью динамической проверки правильности.