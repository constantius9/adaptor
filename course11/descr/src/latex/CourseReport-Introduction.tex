\section{Введение}


\subsection{Мотивация}
Ниже перечислены основные причины заинтересованности в данной задаче.
\begin{enumerate}
    \item Многие исследования в области применения машинного обучения в компиляции \cite{Dubach2009PCO,Thomson2009RTT,Stephenson2003MOI} показывают большой потенциал и востребованность решения данной задачи, но плохо систематизированы.
    \item Инженерная задача по созданию инфраструктуры для коллективной оптимизации не решена до сих пор \cite{Fursin2010COP}.
    \item Конечным пользователям компиляторов мешает именно отсутствие инструментария, который помог бы им пользоваться обучаемым компилятором так же, как обычным (интерфейс чёрного ящика).
    \item Производители аппаратного и программного обеспечения, такие как Intel, IBM, ARM заинтересованы в решении такой задачи \cite{IntelExascale}.
\end{enumerate}


\subsection{Цель}
Конечная цель, которой хотелось бы достичь --- разработать систему сбора, систематизации, формализации данных о производительности компилируемых программ в зависимости от настроек компилятора, а также выполняющую функции поддержки базы знаний и обучения. Цели данной системы представлены ниже.
\begin{enumerate}
    \item Предсказание производительности программ.
    \item Оптимизация производительности программ путём изменения настроек компилятора.
\end{enumerate}

На данном, начальном, этапе работы, рассматривается только 1 критерий оптимальности --- производительность, а именно --- полное время выполнения программы.

Однако, сравнивая стоящую перед нами задачу с текущими успехами в области \cite{Fursin2010COP}, обнаруживаем, что достижение данной цели в рамках курсового проекта не представляется возможным. В связи с этим разработка ограничена учебным заданием и продолжение работы будет осуществляться в рамках дипломного проекта.

Итак, цель данной работы --- разработать основные компоненты расширяемого инструментария для проведения экспериментов по оптимизации программ (в ручном и автоматическом режиме) и прозрачного использования данных из экспериментов в промышленной компиляции.

\subsection{Существующие работы в области}
Как было сказано выше, многие исследования в области применения машинного обучения в компиляции \cite{Dubach2009PCO,Thomson2009RTT,Stephenson2003MOI} показывают большой потенциал и востребованность решения данной задачи, но плохо систематизированы. Вследствие этого основной проблемой является реализация общей базы данных программ и экспериментов и каркаса инструментария для проведения исследований. Реализация всех конечных сценариев, необходимых в исследовании, не является приоритетной задачей, но должно быть осуществимо достаточно простым способом силами самих программных инженеров-конечных пользователей системы и других исследователей.

Ниже перечислены некоторые отличия данной работы от существующих.
\begin{enumerate}
    \item Исследователям и инженерам нужен в первую очередь минимальный функционал для решения описываемой задачи. Графический интерфейс не является необходимым. В сосредоточенности на традиционном текстовом командном интерфейсе основное отличие этой работы от работы \cite{Fursin2010COP}.
    \item Инструментарий не может реализовывать все необходимые сценарии работы с самого начала. Для упрощения разработки сторонними инженерами система должна быть хорошо поддерживаемой и расширяемой. В этом также одно из отличий от работы \cite{Fursin2010COP}.
    \item Разработка системы сфокусирована на применении локально в изоляции от других исследователей, поскольку это --- первоначально необходимый сценарий использования. Разработка многопользовательских сценариев является следующим шагом. Однако используемая технологическая база позволяет относительно просто расширить инструментарий для использования в распределённых системах.
\end{enumerate}

\subsection{Задачи}
Задачи, решаемые в данной работе, перечислены далее.
\begin{enumerate}
    \item Установить репозиторий исходных кодов программ.
    \item Создать и настроить базу данных для хранения экспериментов.
    \item Реализовать интерактивное текстовое окружение для проведения экспериментов, обработки и сохранения данных.
    \item Реализовать построение графиков для визуального представления данных экспериментов.
\end{enumerate}