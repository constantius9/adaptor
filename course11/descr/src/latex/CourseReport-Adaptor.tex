\section{Инструментарий коллективной оптимизации программ}
% TODO


\subsection{Расширяемость инструментария}
Инструментарий должен обеспечивать расширяемость в смысле возможности добавления новых модулей сборки, запуска и анализа данных. Это должно обеспечиваться описанием взаимодействия различных модулей системы. На данный момент эта задача не решается, поскольку более важным является получение результатов хотя бы на ограниченном множестве программ с фиксированным набором модулей сборки, запуска и анализа. На данный момент введение обобщённой архитектуры является преждевременным.


\subsection{Эксперимент по оптимизации программы}
Эксперимент состоит в следующем:

\begin{enumerate}
	\item Система собирант программу с определёнными настройками сборки. В случае компилятора gcc они включают в себя \cite{gcc-options}:

	\begin{enumerate}
		\item компилятор (команда для запуска);
		\item базовый уровень оптимизации (флаг '-O[n]', где n --- уровень оптимизации);
		\item набор флагов тонких настроек оптимизации (флаги семейства '-f[name]', где name --- имя определённого набора оптимизаций). Некоторые из этих флагов также имеют числовые параметры;
		\item путь к заголовочным файлам (опция '-I');
		\item путь к исходным файлам;
		\item параметры определения макросов (флаги вида '-D[MACRO]', где MACRO --- имя определяемого макроса);
		\item путь к исполняемому файлу, производимому компилятором.
	\end{enumerate}

	\item Система запускает программу в контролируемом окружении с определёнными настройками запуска и производит измерение интересующих метрик исполнения программы. Они могут включать в себя:

	\begin{enumerate}
		\item полное время исполнения программы;
		\item время исполнения программы по функциям;
		\item полный объём памяти, занимаемой программой и данными;
		\item объём памяти, занимаемой кодом программы;
	\end{enumerate}

	На данный момент нас будет интересовать полное время исполнения.
	Настройки включают в себя:

	\begin{enumerate}
		\item путь к исполняемому файлу;
		\item источник ввода для стандартного потока ввода программы (перенаправление stdin);
		\item потоки вывода для стандартного потока вывода и стандартного потока ошибок программы (перенаправление stdout и stderr соответственно);
		\item аргументы запуска, в т.ч. путь к обрабатываемому файлу данных (например, путь к файлу изображения для программы сжатия изображений).

	\end{enumerate}

	\item Система сохраняет данные о сборке и запуске программы в базу данных для последующего анализа и обработки.
\end{enumerate}
