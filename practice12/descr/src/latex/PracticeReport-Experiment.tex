\section{Экспериментальная часть}

\section{Методология исследования}

Целью исследования является выявление зависимости между характеристиками аппаратного обеспечения и производительностью программы решения СЛАУ. Анализ должен производиться без знания внутренней организации программы (используемого языка программирования, реализованного метода решения и т.д.), поскольку он должен быть автоматизирован и должен учитывать только свойства, присутствующие в модели производительности программы. На данный момент единственная характеристика задачи, присутствующая в модели --- это размер матрицы, задаваемый числом строк и столбцов.

Выдвинем следующую гипотезу: 

на производительность программы будет влиять размер обрабатываемой матрицы, размер кэша процессора и, в меньшей степени, частота процессора. (1)

Объём оперативной памяти не оказывает влияния до тех пор, пока задача целиком помещается в памяти, поэтому это свойство компьютера не учитывается в модели. Данная гипотеза будет подтверждена или опровергнута путём проведения ранжирования признаков (\textit{англ. feature ranking}) \cite{feature-ranking}.

Выдвинем ещё одну гипотезу:

возможно предсказание прозводительности программы на данной аппаратной платформе при обработке задачи данной размерности. (2)

Эта гипотеза будет проверяться проведением регрессионного анализа и последующего предсказания регрессионной моделью ожидаемой производительности. При этом будет применена методология перекрёстной проверки, как описано в разделе \ref{cross-validation}.

Ввиду ограниченных временных и материальных ресурсов, ограничимся запуском программы на трёх процессорах фирмы Intel:  Core 2 Quad Q8200 с частотой 2,33 ГГц и кэшем третьего уровня объёма 2 МБ, Xeon E5430 с частотой 2,66 ГГц и кэшем третьего уровня объёма 6 МБ, и Core i5 M 460 с частотой 2,53 ГГц и кэшем третьего уровня объёма 3 МБ. Эти процессоры принадлежат к разным семействам (для настольных компьютеров, для серверов и для ноутбуков, соответственно) и выпущены в разное время (2007, 2008, 2010 годы соответственно).

Произведём достаточно много экспериментов на каждом компьютере --- по 1000 запусков. Каждый запуск производится с случайно выбранным размером матрицы. Случайные величины числа столбцов и числа строк матрицы равномерно распределены в диапазоне [2:1024], следовательно, размер матрицы колеблется от $2 \times 2$ до $1024 \times 1024$.

Данный сценарий исследования называется случайным перебором размера матрицы. Он реализован в описываемой системе проведения исследований, как указано в разделе \ref{random-exploration}.

Измерение времени исполнения исследуемой программы производится с использованием процедуры калибровки, как описано в записке \cite{adaptor}. Принципиально, процедура заключается в многократном запуске программы с увеличением числа запусков до тех пор, пока точность измерения времени одного запуска, производимое путём деления общего времени исполнения на число запусков, не достигнет необходимой величины.
