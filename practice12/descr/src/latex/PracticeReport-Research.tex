\section{Исследовательская часть}

\subsection{Моделирование производительности программ}

\subsection{Библиотеки и среды машинного обучения для языка Python}

\subsubsection{Рассмотрение альтернатив}

\begin{itemize}

\item Orange \cite{orange}. Является интегрированной средой для проведения исследований в области анализа данных и машинного обучения и включает в себя библиотеку машинного обучения. Имеет графический интерфейс и предоставляет возможности реализовывать конвейер обработки данных в виде графической блок-схемы, что представляется удобным. Использование среды не требует программирования. Возможно использование входящей в состав системы библиотеки отдельно от интегрированный среды через API \cite{api} для языка программирования Python. Имеет документацию удовлетворительного качества, не содержащую примеров и плохо интегрированную в среду для проведения исследований. Предоставляет обширные возможности по автоматической предварительной обработке данных и визуализации результатов. Интегрированная среду имеет недостаточно высокую стабильность работы. Установку необходимо производить вручную из исходных кодов.
\item sklearn \cite{sklearn}. Является библиотекой машинного обучения. Имеет качественную, достаточно полную, документацию с большим количеством примеров. Удобна в использовании, поскольку интерфейс всех компонентов стандартизирован. Имеет ограниченные возможности визуализации, требующие ручной настройки посредством обращения к API matplotlib \cite{matplotlib}. Является популярной и хорошо протестированной библиотекой. Установка производится автоматически с помощью пакетного менеджера.

\end{itemize}

\subsubsection{Итоговое решение}

В качестве первичного средства проведения обработки и анализа данных будет использоваться среда Orange, поскольку в ней легко производить предварительную обработку и визуализацию данных без необходимости непосредственно программировать, что облегчает работу исследователя.
В качестве дополнительного средва анализа будет использоваться sklearn, поскольку эта библиотека более стабильно работает при использовании сложных методов анализа и предоставляет более удобный интерфейс проведения перекрёстной проверки \cite{cross-validation} и сеточного поиска \cite{grid-search}.

\subsection{Централизованное хранилище экспериментов}

\subsubsection{Рассмотрение альтернатив}

\begin{itemize}

\item IrisCouch. Предоставляет базовые услуги размещения базы данных CouchDB \cite{couchdb}. Находится в режиме бета-тестирования. Использование в пределах определённого объёма является бесплатным \cite{iriscouch}.
\item CloudAnt. Предоставляет услуги размещения базы данных CouchDB \cite{couchdb} с возможностями поиска, RESTful API \cite{restful-api} и интерфейсом Futon \cite{futon}. Готов к промышленному использованию. Использование в пределах определённого объёма является бесплатным \cite{cloudant}.

\end{itemize}

\subsubsection{Итоговое решение}

В качестве хранилища выбран CloudAnt в связи с большей стабильностью и более широкой функциональностью.