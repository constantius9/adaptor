\section{Конструкторская часть}

\subsection{Эксперимент по оценке производительности программы}

Эксперимент по оценке производительности программы представляет собой последовательность шагов, перечисленную ниже.

\begin{enumerate}
	\item Сборка программы.
	\item Измерение времени исполнения программы.
	\item Сохранение данных об эксперименте в базу данных.
\end{enumerate}

Далее, если не указано иное, мы будем называть эксперимент по оценке производительности программы просто «экспериментом».

\subsection{Реализация модели производительности программы}
\label{perf_mod_impl}

Модель производительности программы строится по множеству экспериментов. В системе эксперимент характеризуется свойствами, приведёнными ниже. Многие свойства не являются непосредственно ценными для анализа производительности на данный момент, но сохранены в структуре эксперимента с целью полноты описания и возможности воспроизведения эксперимента другим исследователем --- потенциально на другом компьютере.

\begin{itemize}
	\item Название программы.
	\item Настройки сборки. Включают в себя следующие свойства:
	\begin{itemize}
	    \item используемый компилятор. Задаётся в виде команды запуска компилятора из командной строки;
	    \item базовый уровень оптимизации. Задаётся в виде строки типа «-O<число>», где <число> означает уровень оптимизации. Допустимы значения от 1 до 3;
	    \item дополнительные настройки оптимизации. Задаются в виде строки «-f<имя настройки>», например «-floop-unroll»;
	    \item дополнительные настройки компиляции. Задаются в виде строки, содержащей настройки путей поиска заголовочных файлов («-I<путь>») и библиотек(«-L<путь>»);
	    \item дополнительные настройки связывания. Задаются в виде строки, содержащей список необходимых библиотек («-l<имя библиотеки>»);
	    \item каталог с исходными текстами тестовых программ;
	    \item путь к главному исходному файлу тестовой программы.
	\end{itemize}

	\item Результат запуска с калибровкой. Включает в себя следующие свойства:
	\begin{itemize}
		\item суммарное время исполнения программы;
		\item число запусков программы;
		\item вычисленное время однократного исполнения программы;
		\item дисперсия суммарного времени исполнения;
		\item вариация суммарного времени исполнения;
		\item список суммарных времён исполнения.
	\end{itemize}

	\item Результат проверки. Включает в себя следующие свойства:
	\begin{itemize}
		\item измеренное время однократного исполнения программы;
		\item действительное время однократного исполнения программы (если заведомо известно);
		\item абсолютная ошибка измерения (если известно действительное время однократного исполнения);
		\item относительная ошибка измерения (если известно действительное время однократного исполнения).
	\end{itemize}

	\item Информация об аппаратном обеспечении, на котором исполняется экспериментальная программа. Включает в себя следующую информацию о процессоре:
	\begin{itemize}
		\item текстовое название процессора;
		\item частота процессора;
		\item размер кэша процессора;
		\item список возможностей процессора.
	\end{itemize}

	\item Информация о вычислительной задаче. На данный момент включает в себя два свойства:
	\begin{itemize}
		\item число столбцов в обрабатываемой матрице;
		\item число строк в обрабатываемой матрице.
	\end{itemize}
\end{itemize}

\subsection{Анализ производительности программы}

Ниже приведены непосредственно важные для анализа производительности свойства эксперимента. Это множество свойств является подмножеством всех свойств эксперимента, описанных в разделе \ref{perf_mod_impl}.

\begin{itemize}
	\item Название программы.
	\item Используемый компилятор.
	\item Используемый базовый уровень оптимизации.
	\item Используемые дополнительные настройки оптимизации.
	\item Число строк в обрабатываемой матрице.
	\item Число столбцов в обрабатываемой матрице.
	\item Измеренное время однократного исполнения программы.
	\item Название процессора, на котором исполнялась программа.
	\item Частота процессора.
	\item Размер кэша процессора.
	\item Множество дополнительных возможностей, поддерживаемых или не поддерживаемых процессором. Примером такого свойства является наличие поддержки процессором набора инструкций SSE \cite{sse}.
\end{itemize}


\subsection{Централизованное хранилище экспериментов}

Для обмена данными с централизованным хранилищем экспериментов реализован модуль работы с базой данных CouchDB. Он основан на библиотеке couchdbkit \cite{couchdbkit} для языка программирования Python и реализует документы (классы сущностей, хранимых в БД), описанные ниже. Документы хранят свойства соответствующих сущностей в формате, пригодном для хранения в БД.

\begin{itemize}

\item Документ эксперимента.
\item Документ результата измерения времени исполнения программы.
\item Документ результата проверки точности измерения времени исполнения программы.
\item Документ информации об аппаратном обеспечении.
\item Документ настроек сборки программы.

\end{itemize}