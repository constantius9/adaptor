\section{Конструкторская часть}

За основу взята система, разработанная в рамках курсового проекта \cite{adaptor}. Представление эксперимента доработано так, как описано в следующем разделе.

\subsection{Структура набора тестовых программ Polybench}

Структура каталогов с исходными кодами тестовых программ представляет собой дерево. Типичный путь до папки исходных кодов выглядит так: 'polybench/linear-algebra/kernels/symm'.

Для обеспечения простоты использования инструментария был реализован поиск исходных кодов заданной программы в дереве каталогов. При начале работы с системой пользователь лишь задаёт название программы, которую он хочет исследовать, а инструментарий находит исходные коды автоматически.

Для этого был реализован поиск по дереву путей путём обхода в ширину. Поиск останавливается, когда имя каталога, имеющего наибольший уровень вложенности, совпадает с заданным пользователем названием программы.

\subsection{Эксперимент по оценке производительности программы}

Эксперимент по оценке производительности программы представляет собой последовательность шагов, перечисленную ниже.

\begin{enumerate}
	\item Сборка программы.
	\item Измерение времени исполнения программы.
	\item Сохранение данных об эксперименте в базу данных.
\end{enumerate}

Далее, если не указано иное, мы будем называть эксперимент по оценке производительности программы просто «экспериментом».

\subsection{Реализация модели производительности программы}
\label{perf_mod_impl}

Модель производительности программы строится по множеству экспериментов. В системе эксперимент характеризуется свойствами, приведёнными ниже. Многие свойства не являются непосредственно ценными для анализа производительности на данный момент, но сохранены в структуре эксперимента с целью полноты описания и возможности воспроизведения эксперимента другим исследователем --- потенциально на другом компьютере.

\begin{itemize}
	\item Название программы.
	\item Настройки сборки. Включают в себя следующие свойства:
	\begin{itemize}
	    \item используемый компилятор. Задаётся в виде команды запуска компилятора из командной строки;
	    \item базовый уровень оптимизации. Задаётся в виде строки типа «-O<число>», где <число> означает уровень оптимизации. Допустимы значения от 1 до 3;
	    \item дополнительные настройки оптимизации. Задаются в виде строки «-f<имя настройки>», например «-floop-unroll»;
	    \item дополнительные настройки компиляции. Задаются в виде строки, содержащей настройки путей поиска заголовочных файлов («-I<путь>») и библиотек(«-L<путь>»);
	    \item дополнительные настройки связывания. Задаются в виде строки, содержащей список необходимых библиотек («-l<имя библиотеки>»);
	    \item каталог с исходными текстами тестовых программ;
	    \item путь к главному исходному файлу тестовой программы.
	\end{itemize}

	\item Результат запуска с калибровкой. Включает в себя следующие свойства:
	\begin{itemize}
		\item суммарное время исполнения программы;
		\item число запусков программы;
		\item вычисленное время однократного исполнения программы;
		\item дисперсия суммарного времени исполнения;
		\item вариация суммарного времени исполнения;
		\item список суммарных времён исполнения.
	\end{itemize}

	\item Результат проверки. Включает в себя следующие свойства:
	\begin{itemize}
		\item измеренное время однократного исполнения программы;
		\item действительное время однократного исполнения программы (если заведомо известно);
		\item абсолютная ошибка измерения (если известно действительное время однократного исполнения);
		\item относительная ошибка измерения (если известно действительное время однократного исполнения).
	\end{itemize}

	\item Информация об аппаратном обеспечении, на котором исполняется экспериментальная программа. Включает в себя следующую информацию о процессоре:
	\begin{itemize}
		\item текстовое название процессора;
		\item частота процессора;
		\item размер кэша процессора;
		\item список возможностей процессора.
	\end{itemize}

	\item Информация о вычислительной задаче. На данный момент включает в себя два свойства:
	\begin{itemize}
		\item число столбцов в обрабатываемой матрице;
		\item число строк в обрабатываемой матрице.
	\end{itemize}
\end{itemize}

\subsection{Анализ производительности программы}

Ниже приведены непосредственно важные для анализа производительности свойства эксперимента. Это множество свойств является подмножеством всех свойств эксперимента, описанных в разделе \ref{perf_mod_impl}.

\begin{itemize}
	\item Название программы.
	\item Используемый компилятор.
	\item Используемый базовый уровень оптимизации.
	\item Используемые дополнительные настройки оптимизации.
	\item Число строк в обрабатываемой матрице.
	\item Число столбцов в обрабатываемой матрице.
	\item Измеренное время однократного исполнения программы.
	\item Название процессора, на котором исполнялась программа.
	\item Частота процессора.
	\item Размер кэша процессора.
	\item Множество дополнительных возможностей, поддерживаемых или не поддерживаемых процессором. Примером такого свойства является наличие поддержки процессором набора инструкций SSE \cite{sse}.
\end{itemize}

\subsection{Централизованное хранилище экспериментов}

\subsubsection{Модуль работы с базой данных}

Для обмена данными с централизованным хранилищем экспериментов реализован модуль работы с базой данных CouchDB. Он основан на библиотеке couchdbkit \cite{couchdbkit} для языка программирования Python и реализует документы (классы сущностей, хранимых в БД), описанные ниже. Документы хранят свойства соответствующих сущностей в формате, пригодном для хранения в БД. Состав документа индентичен составу соотвествующей сущности, как описано в предыдущем разделе.

\begin{itemize}

\item Документ эксперимента.
\item Документ результата измерения времени исполнения программы.
\item Документ результата проверки точности измерения времени исполнения программы.
\item Документ информации об аппаратном обеспечении.
\item Документ настроек сборки программы.

\end{itemize}

\subsubsection{Размещение базы данных на удалённом сервере}

В качестве удалённого сервера выбран сервер Amazon Elastic Computing (иначе называемый Amazon EC2) \cite{amazon-ec2}. Amazon EC2 предоставляет услуги размещения ресурсов в интернет по схеме так называемых «облачных вычислений» (\textit{англ. cloud computing} \cite{cloud-computing}). Получение доступа к серверу, установка на него операционной системы, включение и выключение, подсчёт стоимости автоматизированы и выполняются через веб-интерфейс. Стоимость предоставления услуг является очень низкой при текущих объёмах использования вычислительных мощностей, хранилища и передаваемых через сеть данных \cite{amazon-billing}. Объём предоставляемых услуг легко поддаётся масштабированию при увеличении нагрузки.

Для обеспечения возможностей централизованного сбора данных и их анализа в фоновом режиме (\textit{англ. offline}) был реализован серверный модуль для работы на платформе Amazon EC2. Данный модуль может запускать сценарии анализа накопленных экспериментов по расписанию или при наступлении каких-либо других условий --- например, по достижении определённого количества сохранённых в БД экспериментов. Это необходимо для постоянного улучшения модели производительности программ на основе поступающих данных.

Серверный модуль также может выполнять эксперименты непосредственно на аппаратном обеспечении сервера. На данный момент это полезно и допустимо, поскольку нагрузка на сервер, связанная с обработкой клиентских запросов, пренебрежимо мала, а данные о запусках программ на ещё одной аппаратной платформе имеют большую ценность в связи с ограниченными материальными ресурсами.

Данный модуль также поддерживает работу как с удалённой БД на платформе CloudAnt, так и с локальной (относительно сервера) базой данных. А именно, база данных CouchDB может быть установлена на том же физическом сервере Amazon EC2, что обеспечивает более низкие задержки передачи данных между серверным модулем и БД. Переключение испоьзуемой БД производится изменением одного параметра настройки системы.

Отдельно следует отметить высокую совместимость клиентского и серверного модулей инструментария. Оба они могут работать как с локальной, так и с удалённой БД. Оба модуля поддерживают одно и то же множество сценариев и предоставляют пользователю одинаковые команды. Возможна двусторонняя синхронизация данных между базами данных CouchDB, расположенными на разных физических машинах.

Перспективной является возможность реализации API для работы с системой как с веб-сервисом. На данный момент API не реализовано ввиду отсутствия необходимости. Наличие API имеет смысл при большом спросе на доступ к системе со стороны клиентов.

Сочетание использования сервера Amazon EC2 и базы данных CouchDB делает систему легко масштабируемой и позволяет легко построить веб-сервис по модели «облачных вычислений».

\subsection{Сценарии исследования}

Все сценарии исследования подразумевают многократный запуск одной и той же программы на одном компьютере с разными размерами входных данных. Затем сценарий повторяется на всех интересующих исследователя компьютерах, в процессе чего происходит накопление экспериментов в общей базе данных.

\subsubsection{Логарифмический перебор}

В данном сценарии размер входных данных задаётся числом $N$ из заданного диапазона, причём количество строк и столбцов обрабатываемой матрицы принимаются одинаковыми. Таким образом, этот сценарий рассматривает только обработку квадратных матриц. Число $N$ меняется по закону $N = 2 ^ i$, где $i$ --- номер эксперимента.

\subsubsection{Случайный однообразный перебор}
В данном сценарии размер входных данных задаётся числом $N$ из заданного диапазона, означающим количество строк и столбцов в обрабатываемой матрице. Число генерируется случайным образом и распределено по диапазону равномерно. Этот сценарий рассматривает только обработку квадратных матриц. 

\subsubsection{Случайный перебор}
\label{random-exploration}

В данном сценарии размер входных данных задаётся парой чисел $N$ и $M$ из заданного диапазона, означающих количество строк и столбцов в обрабатываемой матрице, соответственно. Оба числа генерируются случайным образом и распределены по диапазону равномерно.

\subsection{Процесс работы системы в фоновом режиме}

В случае запуска системы на сервере возможна её работа в фоновом режиме. В таком случае алгоритм работы выглядит следующим образом.

\begin{enumerate}
\item Начало итерации. Пока не накоплено достаточное количество экспериментов в БД, ничего не делать. На этом шаге система находится в пассивном состоянии, ожидая ввода со стороны клиентов. Клиенты производят эксперименты в обычном режиме, при этом сохраняя их в удалённую БД, за которой следит сервер. Как только накоплено достаточное количество экспериментов, система переходит к следующему шагу.
\item Произвести обучение модели производительности программ на имеющихся экспериментах.
\item Произвести перекрёстную проверку модели на имеющихся данных. Если точность модели по результатам перекрёстной проверки ниже определённой константы, перейти к следующему шагу. Иначе, перейти к шагу 1 и закончить итерацию.
\item Произвести ранжирование признаков модели и отобрать на 1 признак больше, чем было включено в модель на предыдущей итерации. Перейти на шаг 2.
\end{enumerate}

Данный алгоритм обеспечивает периодическую подстройку модели под новые данные. Он является адаптивным и со временем начинает использовать всё больше и больше признаков экспериментов для увеличения точности модели. Его слабой стороной является требование наличия большого числа релевантных признаков, из которых можно выбирать на шаге 4. В данном случае система опирается на работу исследователей по добавлению модулей сбора всё новых и новых признаков.

При добавлении нового признака в модель изменение базы данных уже имеющихся экспериментов не требуется. Поскольку CouchDB является документо-ориентированной БД, она не требует наличия жёсткой схемы и новые поля могут добавляться в документы динамически. При этой наобходимо добавление кода в модуль связи с БД. Этот код должен отслеживать обращение к документам и в случае отсутствия необходимых полей (например, если получен документ старой версии, в котором ещё не было нового признака) производить действия по дополнению документов недостающими данными с помощью эмпирических правил.

\subsection{Предварительная обработка данных для анализа}

Система включает модуль предварительной обработки данных. Этот модуль связывается с БД, запрашивает документы, отвечающие определённому критерию, производит преобразование полученных документов в формат Comma Separated Value (CSV) и сохраняет их в файл.

\subsubsection{Критерии получения экспериментов из БД}

Для ограничения области изучения возможно получение из БД только тех документов, которые представляют интерес для исследователя в данный момент времени. Фильтрация документов возможна по следующим свойствам:

\begin{itemize}
\item дата и время проиведения эксперимента;
\item имя исследуемой программы;
\item название серии экспериментов (задаётся исследователем перед проведением серии экспериментов и используется для группировки).
\end{itemize}

Эти варианты доступа к экспериментам реализованы в виде представлений (\textit{англ. view}) в приложении CouchApp \cite{couchapp}. Представления программируются на языке JavaScript и представляют собой правила отбора документов по нужным критериям. Ключевой особенностью представлений CouchApp является то, что, будучи однажды построенным, представление производит очень быстрый отбор данных. В данном смысле оно подобно индексу в традиционно реляционной базе данных.
