\section{Введение}
\subsection{Цель}

Цель данной работы --- произвести анализ производительности программ из набора Polybench с целью построения моделей производительности, а также разработать способы автоматизации построения и улучшения моделей.

\subsection{Задачи}
Задачи, решаемые в данной работе, перечислены далее в примерном порядке выполнения.
\begin{enumerate}
	\item Построить первоначальную образцовую модель производительности. Эта модель будет простейшей в смысле описываемой зависимости времени выполнения от других свойств программы, набора данных и платформы -- она будет описывать зависимость только от одного свойства. Возможно построение многих таких моделей с целью нахождения подходящей для анализа и расширения. Расширение модели будет производиться для увеличения точности, с которой она описывает поведение реальной программы в реальном окружении.
	\item Выявить недостатки первоначальной модели. Поскольку простейшая модель наверняка не сможет показать практически полезных результатов, необходимо выявить её недостатки и предложить способы их устранения.
	\item Разработать программу анализа модели производительности с целью выявления недостаточности имеющихся в модели признаков. Эта программа должна суметь проанализировать модель с целью определения того, является ли она адекватной -- т.е. хорошо описывающей поведение реальной программы.
	\item Произвести добавление в модель признаков, позволяющих более точно описать производительность программы на данной платформе, в ручном или автоматическом режиме. Это производится для улучшения качества описания поведения моделью производительности. Это будет произведено в ручном режиме с последующей попыткой запрограммировать введение новых признаков. Новые признаки позволят описать более сложную зависимость производительности от свойств исследуюемой системы.
	\item Произвести оценку разработанной программы на наборе моделей, полученных с помощью анализа программ, входящих в состав набора Polybench. Для проверки работы инструментария требуется построить и проанализировать несколько разных моделей. Затем необходимо рассмотреть возникшие проблемы.
\end{enumerate}