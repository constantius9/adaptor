\section{Экономически-организационная часть}
\subsection{Введение}
Экономическая часть посвящена разработке комплекса мероприятий организационно–экономического и финансового планов, которые необходимо выполнить для создания программного продукта, позволяющего проводить анализ и моделирование производительности программ.

Система разработана на языке программирования Python, и включает в себя группу проектов анализа данных в системе проведения статистических исследований Orange. Программный продукт не может быть интегрирован в другие программные комплексы и предназначен для внутреннего использования. Заказчиком является структурное подразделение МГТУ им. Н. Э. Баумана.
Данный проект не преследует цель получения прибыли.

\subsection{Основные этапы проекта разработки нового изделия}
Основные этапы проекта разработки нового изделия представлены в таблице \ref{tab:development-stages}.

\begin{table}[H]
    \caption{\label{tab:development-stages}Основные этапы разработки проекта}
    \begin{tabular}[H]{|l|p{5cm}|p{8cm}|}
        \hline
        № & Условное обозначение & Описание\\
        \hline
        1 & Техническое задание (ТЗ) & Предпроектное обследование. Постановка задачи по созданию программного продукта\\
        \hline
        2 & Эскизный проект (ЭП) & Проработка использования основных технологий и инструментов, необходимых для выполнения программного продукта\\
        \hline
        3 & Технорабочий проект (ТП) & Разработка модели анализа: структуры данных и реализации алгоритмов анализа. Разработка программного решения\\
        \hline
        4 & Документация и внедрение (В) & Подготовка и передача программного продукта и программной документации\\
        \hline
    \end{tabular}
\end{table}

\subsection{Расчёт трудоёмкости разработки программного продукта}
Трудоёмкость разработки программного продукта рассчитывается по формуле
\begin{equation*}
\tau_{\pe\pe} = \tau_{\te\ze} + \tau_{\erev\pe} + \tau_{\te\pe} + \tau_{\re\pe} + \tau_{\ve}
\end{equation*}