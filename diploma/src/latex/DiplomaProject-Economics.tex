\section{Экономически-организационная часть}
\subsection{Введение}
Экономическая часть посвящена разработке комплекса мероприятий организационно–экономического и финансового планов, которые необходимо выполнить для создания программного продукта, позволяющего проводить анализ и моделирование производительности программ.

Система разработана на языке программирования Python, и включает в себя группу проектов анализа данных в системе проведения статистических исследований Orange. Программный продукт не может быть интегрирован в другие программные комплексы и предназначен для внутреннего использования. Заказчиком является структурное подразделение МГТУ им. Н. Э. Баумана.
Данный проект не преследует цель получения прибыли.

\subsection{Основные этапы проекта разработки нового изделия}
Основные этапы проекта разработки нового изделия представлены в таблице \ref{tab:development-stages}.

\begin{table}[H]
    \caption{\label{tab:development-stages}Основные этапы разработки проекта}
    \begin{tabular}[H]{|l|p{5cm}|p{8cm}|}
        \hline
        № & Условное обозначение & Описание\\
        \hline
        1 & Техническое задание (ТЗ) & Предпроектное обследование. Постановка задачи по созданию программного продукта\\
        \hline
        2 & Эскизный проект (ЭП) & Проработка использования основных технологий и инструментов, необходимых для выполнения программного продукта\\
        \hline
        3 & Технорабочий проект (ТП) & Разработка модели анализа: структуры данных и реализации алгоритмов анализа. Разработка программного решения\\
        \hline
        4 & Документация и внедрение (В) & Подготовка и передача программного продукта и программной документации\\
        \hline
    \end{tabular}
\end{table}

\subsection{Расчёт трудоёмкости разработки программного продукта}
Трудоёмкость разработки программного продукта рассчитывается по формуле
\begin{equation*}
	\tau_{\pe\pe} = \tau_{\te\ze} + \tau_{\erev\pe} + \tau_{\te\pe} + \tau_{\re\pe} + \tau_{\ve},
\end{equation*}
где $\tau_{\te\ze}$ -- трудоёмкость разработки технического задания на создание ПП; $\tau_{\erev\pe}$ -- трудоёмкость разработки эскизного проекта ПП; $\tau_{\re\pe}$ -- трудоёмкость разработки рабочего проекта ПП; $\tau_{\ve}$ -- трудоёмкость внедрения разработанного ПП.

Трудоёмкость разработки технического задания можно рассчитать по формуле
\begin{equation*}
	\tau_{\te\ze} = T_{\ze\re\ze} + T_{\ze\re\pe}
\end{equation*}
где $T_{\ze\re\ze}$ -- затраты времени разработчика постановки задачи на разработку ТЗ, человеко-дни, $T_{\ze\re\pe}$ -- затраты времени разработчика программного обеспечения на разработку ТЗ, человеко-дни.

Значения затрат времени $T_{\ze\re\ze}$ и $T_{\ze\re\pe}$ определяют по формулам:
\begin{eqnarray*}
	T_{\ze\re\ze} = t_{\ze} K_{\ze\re\ze} \\
	T_{\ze\re\pe} = t_{\ze} K_{\ze\re\pe},
\end{eqnarray*}
где $t_{\ze}$ -- норма времени на разработку ТЗ на программный продукт в зависимости от функционального назначения и степени новизны разрабатываемого ПП, человеко-дни;

$K_{\ze\re\ze}$ -- коэффициент, учитывающий удельный вес трудоемкости работ, выполняемых разработчиком постановки задач на стадии ТЗ (в случае совместной с разработчиком ПО разработки $K_{\ze\re\ze}=0,65$, в случае самостоятельной разработки ТЗ $K_{\ze\re\ze}=1$); 
$K_{\ze\re\pe}$ -- коэффициент, учитывающий удельный вес трудоемкости работ, выполняемых разработчиком ПП на стадии ТЗ (в случае совместной с постановки задач разработки $K_{\ze\re\pe} = 0,35$, в случае самостоятельной разработки ТЗ $K_{\ze\re\pe} = 1$).

В данном случае следует принять следующие значения коэффициентов
\begin{equation}
	t_{\ze} = 3 \che\e\el.-\de\en\e\ishrt.
\end{equation}

Тогда трудоёмкость разработки технического задания
\begin{equation}
	\tau_{\te\ze} = T_{\ze\re\ze} + T_{\te\re\pe} = 3 \cdot 1 = 3 \che\e\el.-\de\en\e\ishrt.
\end{equation}

Трудоёмкость разработки эскизного проекта ПП рассчитывают по формуле
\begin{equation}
	\tau_{\erev\pe} = T_{\erev\re\ze} + T_{\erev\re\pe},
\end{equation}
где $T_{\erev\re\ze}$ -- затраты времени разработчика постановки задачи на разработку эскизного проекта, человеко-дни, $T_{\erev\re\pe}$ -- затраты времени разработчика ПП на разработку эскизного проекта, человеко-дни.

Значения величин $T_{\erev\re\ze}$ и $T_{\erev\re\pe}$ рассчтываются по формулам:
\begin{eqnarray*}
	T_{\erev\re\pe} = t_{\erev} K_{\erev\re\pe}
	T_{\erev\re\ze} = t_{\erev} K_{\erev\re\ze}, \\
\end{eqnarray*}

где $t_{erev}$ -- норма времени на разработку эскизного проекта на ПП в зависимости от функционального назначения и степени новизны разрабатываемого ПП, человеко-дни;

$K_{\erev\re\pe}$ -- коэффициент, учитывающий удельный вес трудоёмкости работ, выполняемых разработчиком постановки задач на стадии эскизного проекта;

$K_{\erev\re\ze}$ -- коэффициент, учитывающий удельный вес трудоёмкости работ, выполняемых разработчиком ПП на стадии эскизного проекта.

В данном случае следует принять следующие значения коэффициентов
\begin{equation*}
	t_{\erev} = 18 \che\e\el.-\de\en\e\ishrt, K_{\erev\re\pe} = 0, K_{\erev\re\ze} = 1.
\end{equation*}
Тогда трудоемкость разработки эскизного проекта ПП равна
\begin{equation*}
	\tau_{\erev\pe} = T_{\erev\re\ze} + T_{\erev\re\pe} = 18 \che\e\el.-\de\en\e\ishrt.
\end{equation*}

Трудоемкость разработки технического проекта зависит от функционального назначения ПП, количества разновидностей форм входной и выходной информации и определяется как сумма времени, затраченного разработчиком постановки задач и разработчиком программного обеспечения: 
\begin{equation*}
	\tau_{\te\pe} = (t_{\te\re\ze} + t_{\te\re\pe}) \cdot K_{\ve} K_{\re},
\end{equation*}
где $t_{\te\re\ze}$ и $t_{\te\re\pe}$ -- норма времени, затрачиваемого на разработку ТП разработчиком постановки задач и разработчиком ПП соответственно, человеко-дни ($t_{\te\re\pe} = 20\che\e\el.-\de\en\e\ishrt$);

$K_{\ve}$ -- коэффициент учёта вида используемой информации;

$K_{\re}$ -- коэффициент учёта режима обработки информации ($K_{\re} = 1,26$).

Значение коэффициента $K_{\ve}$ определяется по формуле
\begin{equation*}
    K_{\ve} = \frac{K_{\pe}n_{\pe} + K_{\en\es}n_{\en\es} + K_{\be}n_{\be}}{n_{\pe} + n_{\en\es} + n_{\be}}
\end{equation*}
 
где	$K_{\pe}$, $K_{\en\es}$, $K_{\be}$, -- значения коэффициентов учета вида используемой информации для переменной, нормативно–справочной информации и баз данных соответственно;
$n_{\pe}$, $n_{\en\es}$, $n_{\be}$ -- количество наборов данных переменной, нормативно–справочной информации и баз данных соответственно. 
\begin{equation*}
	K_{\ve} = (1,0\cdot4 + 0,72\cdot2 + 2,08\cdot4)/10 = 1,376
\end{equation*}
Тогда трудоемкость разработки технического проекта
\begin{equation*}
	\tau_{\te\pe} = (t_{\te\re\ze} + t_{\te\re\pe}) \cdot K_{\ve} K_{\re} = 20 \cdot 1,26 \cdot 1,376 = 35 \che\e\el.-\de\en\e\ishrt,
\end{equation*}
Трудоемкость разработки рабочего проекта зависит от функционального назначения ПП, количества разновидностей форм входной и выходной информации, сложности алгоритма, сложности контроля информации, степени использования готовых программных модулей, уровня алгоритмического языка программирования и определяется по формуле:
\begin{equation*}
	\tau_{\re\pe} = (t_{\re\re\ze} + t_{\re\re\pe}) \cdot K_{\ka} K_{\re} K_{\ya} K_{\ge} K_{\ci\ca},
\end{equation*}
где $K_{\ka}$ -- коэффициент учета сложности контроля информации ($K_{\ka} = 1,00$);
$K_{\re}$ -- коэффициент учета режима обработки информации ($K_{\re} = 1,32$); 
$K_{\ya}$ -- коэффициент учета уровня используемого алгоритмического языка программирования ($K_{\ya} = 1,00$); 
$K_{\ge}$ -- коэффициент учета степени использования готовых программных модулей -- 25\% – 40\% ($K_{\ge} = 0,70$); 
$K_{\ci\ca}$ -- коэффициент учета вида используемой информации и сложности алгоритма ПП.
Значение коэффициента $K_{\ci\ca}$ определяют по формуле
\begin{equation*}
    K_{\ci\ca} = \frac{K'_{\pe}n_{\pe} + K'_{\en\es}n_{\en\es} + K'_{\be}n_{\be}}{n_{\pe} + n_{\en\es} + n_{\be}}
\end{equation*}
где $K'_{\pe}$, $K'_{\en\es}$, $K'_{\be}$ -- значения коэффициентов учета сложности алгоритма ПП и вида используемой информации для переменной, нормативно–справочной информации и баз данных соответственно, $K_{\ci\ca} = 0,98$;
$t_{\re\re\ze}$, $t_{\re\re\pe}$ -- норма времени, затраченного на разработку РП на алгоритмическом языке высокого уровня разработчиком постановки задач и разработчиком программного обеспечения соответственно, чел.–дни. ($t_{\re\re\pe} = 51$ чел.-дней), так как используется 2 вида входной и 2 вида выходной информации.
 
Соответственно трудоемкость технорабочего проекта можно рассчитать следующим образом:
\begin{equation*}
	\tau_{\te\re\pe}=0,85 \cdot \tau_{\te\pe} + \tau_{\re\pe} = 0,85 \cdot 35 + 50 = 80 \che\e\el.-\de\en\e\ishrt.
\end{equation*}
Трудоемкость выполнения стадии внедрения рассчитывается по формуле:
\begin{equation*}
	t_{\ve} = (t_{\ve\re\ze} + t_{\ve\re\pe}) \cdot K_{\ka} K_{\re} K_{\ze},
\end{equation*}
где $t_{\ve\re\ze}$, $t_{\ve\re\pe}$ -- норма времени, затрачиваемого разработчиком постановки задач и разработчиком программного обеспечения соответственно на выполнение процедур внедрения ПП, человеко–дни. 
\begin{eqnarray*}
	t_{\ve\re\pe} = 10 \che\e\el.-\de\en\e\ishrt, \\
	K_{\ka} = 1,08, K_{\re} = 1,32, K_{\ze} = 0,7.
\end{eqnarray*}
Подставив значения коэффициентов, получим
\begin{equation*}
	t_{\ve} = (t_{\ve\re\ze} + t_{\ve\re\pe}) \cdot K_{\ka} K_{\re} K_{\ze} = 10\cdot1,08\cdot1,32\cdot0,7 \approx 10 \che\e\el.-\de\en\e\ishrt.
\end{equation*}
Таким образом, трудоемкость разработки программной продукции   составляет
 
Планирование и контроль хода выполнения разработки проводят по календарному графику выполнения работ. Можно использовать ленточный график (график Гантта), который представляет собой графическое отображение выполненной работы и времени, затраченного на эту работу, т. е. продолжительность выполнения данной работы.
Продолжительность выполнения всех работ по этапам разработки ПП определяют из выражения
 ,
где  – трудоемкость i – й работы, чел.–дни; 
Q – трудоемкость дополнительных работ, выполняемых исполнителем, чел–дни; 
ni – количество исполнителей, выполняющих i – ю работу, чел;
f2011 – коэффициент пересчета из рб.–дн. в кл.–дн., равный, для 2008 года, 248/365=0,679.
