\section{Охрана труда и экология}
\subsection{Проектирование рабочего места оператора ПЭВМ}
Требования к компьютерной технике и к условиям работы с ней в Российской Федерации регламентируются санитарными нормами и правилами СанПиН 2.2.2/2.4.1340-03 и СанПиН 2.2.2.000-02. Рассмотрим основные нормы, необходимые для проектирования рабочего места оператора ПЭВМ.
\subsubsection{Требования к рабочим помещениям}
Согласно СанПиН 2.2.2/2.4.1340-03, помещения для работы с компьютерами должны оборудоваться системами отопления, кондиционирования воздуха или эффективной приточно-вытяжной вентиляцией. Звукоизоляция помещений и звукопоглощение ограждающих конструкций помещения должны отвечать гигиеническим требованиям и обеспечивать нормируемые параметры шума на рабочих местах. Помещения должны иметь естественное и искусственное освещение.
Поверхность пола в помещениях должна быть ровной, без выбоин, нескользкой, удобной для очистки и влажной уборки, обладать антистатическими свойствами. При строительстве новых и реконструкции действующих средних, средних специальных и высших учебных заведений помещения для работы с компьютером следует проектировать высотой (от пола до потолка) не менее 4,0 м.
Расположение рабочих мест для взрослых пользователей в подвальных помещениях не допускается.
