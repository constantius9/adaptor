\section*{Введение}
В ходе проектирования компьютерных систем часто возникает задача оценки быстродействия определённых программ на данной системе. Традиционным методом решения данной задачи является эмуляция исполнения программы \cite{emulation}. У этого метода есть несколько недостатков. Во-первых, он требует полной реализации поведения эмулируемого компьютера, что требует больших вложений ресурсов \cite{emulation-complexity}. Во-вторых, скорость выполнения программы на эмулируемом компьютере меньше скорости реального выполнения в сотни и тысячи раз \cite{emulation-speed}.

Для решения этих проблем предлагается построить статистическую модель быстродействия программы на исследуемом компьютере.

В данной области проведено большое число исследований. C.Dubach et. al. производили моделирование быстродействия программ для встраиваемых компьютерных систем. Основные методы используемые ими методы "--- это \textit{PCA}, \textit{K--Means}, \textit{SVM}.

W.Wu и B.C.Lee исследовали модели быстродействия программ для персональных компьютеров на основе набора тестовых программ \textit{SPEC2006}. Они использовали эволюционный метод. Достигнутая ошибка предсказания составляет от 8 до 10\%.

В данной работе рассматривается моделирование быстродействия программ для персональных компьютеров на основе набора тестовых программ \textit{Polybench/C}. Основные применяемые методы "--- \textit{kNN }и \textit{Earth Importance}. Достигнута ошибка предсказания около 5\%.

В рамках работы была поставлена цель разработать метод построения статистических моделей быстродействия программ на компьютерах общего назначения, произвести его программную реализацию в составе инструментария анализа быстродействия \textit{Adaptor} и оценить эффективность разработанного метода на избранных программах из набора тестов \textit{Polybench/C}.

Как уже было отмечено выше, одной из областей применения моделей компьютерных систем является оценка быстродействия ещё не произведённого компьютера на этапе его разработки.

Другой пример "--- оценка быстродействия программы при итеративной компиляции.

Работа структурирована следующим образом. В разделе \ref{sec:methodology} мы рассмотрим предложенный метод построения моделей быстродействия программ \textit{Velocitas} и вспомогательный метод измерения времени исполнения программ с повышенной точностью. Далее, в разделе \ref{sec:construction} описана программная реализация метода \textit{Velocitas} в составе инструментария \textit{Adaptor}. В следующем разделе мы приведём результаты оценки эффективности программной реализации метода \textit{Velocitas}. В заключительном разделе мы подводим итоги и говорим о дальнейшей работе.