\section*{Введение}
\subsection*{Обзор области}
В ходе проектирования компьютерных систем часто возникает задача оценки быстродействия определённых программ на данной системе. Традиционным методом решения данной задачи является эмуляция исполнения программы \cite{emulation}. У этого метода есть несколько недостатков. Во-первых, он требует полной реализации поведения эмулируемого компьютера, что требует больших вложений ресурсов \cite{emulation-complexity}. Во-вторых, скорость выполнения программы на эмулируемом компьютере меньше скорости реального выполнения в сотни и тысячи раз \cite{emulation-speed}.
\subsection*{Актуальность}
Как уже было отмечено выше, одной из областей применения моделирования компьютерных систем является оценка быстродействия ещё не произведённого компьютера на этапе его разработки.

Другой пример "--- оценка быстродействия программы при итеративной компиляции.